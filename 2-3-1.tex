%%TODO 縦ベクトルはどう書く?frame
\begin{frame}{条件付きガウス分布(導入)}
\begin{itemize}
 \item 多変量ガウス分布の条件付き分布を考える
       \begin{equation}
				x\sim N(x|\mu, \Sigma)
       \end{equation}
       \begin{equation}
	x = (x_a, x_b)
       \end{equation}
       \begin{equation}
	\mu = (\mu_a, \mu_b)
       \end{equation}

 \item 精度行列$\Lambda$を導入する
 \begin{equation}
	\Lambda \equiv \Sigma^{-1}
 \end{equation}

 \item ベクトル$x$の分割に対応する、分割された形式の精度行列を導入する
       \begin{equation}
				\Lambda=
				 \begin{pmatrix}
					\Lambda_{aa} && \Lambda_{ab}\\
					\Lambda_{ba} && \Lambda_{bb}\\
				 \end{pmatrix}
       \end{equation}
\end{itemize}
\end{frame}

\begin{frame}{やること}
\begin{itemize}
 \item 条件付きガウス分布もガウス分布に従うことを示す
 \item 条件付きガウス分布は、
			 \begin{equation}
				 % p(x_a | x_b) \sim N(x | \mu_{a|b}, \Sigma_{a|b})
 p(x_a | x_b) = \frac{p(x_a, x_b)}{p(x_b)}
			 \end{equation}
 \item $x_b$を観測済の値で固定する
 \item ここで、$p(x_b)$は正規化するための係数
 \item まず$p(x_a,x_b)$の指数部に注目する

\end{itemize}

\end{frame}

\begin{frame}{条件付き分布の表現}
\begin{itemize}
 \item ガウス分布の指数部分に注目する
			 \begin{eqnarray*}
				-\frac{1}{2}(x - \mu)^{T}\Sigma^{-1}(x-\mu) &= &
				 -\frac{1}{2}(x_a - \mu_a)^{T}\Sigma_{aa}^{-1}(x_a-\mu_a) \\
				&&-\frac{1}{2}(x_a - \mu_a)^{T}\Sigma_{ab}^{-1}(x_b-\mu_b) \\
				 &&-\frac{1}{2}(x_b - \mu_b)^{T}\Sigma_{ba}^{-1}(x_a-\mu_a) \\
				 &&-\frac{1}{2}(x_b - \mu_b)^{T}\Sigma_{bb}^{-1}(x_b-\mu_b)
			 \end{eqnarray*}
 \item $x_a$の関数として見ると、これも二次形式になっている
 \item 対応する条件付き分布$p(x_a | x_b)$もガウス分布となる
\end{itemize}
\end{frame}

\begin{frame}{平方完成}
\begin{itemize}
 \item 平方完成とは、○○すること %TODO
\begin{equation}
 -\frac{1}{2}(x-\mu)^{T}\Sigma^{-1}(x-\mu) = -\frac{1}{2}x^T\Sigma^{-1}x+x^T\Sigma^{-1}\mu + const\label{135015_17Nov14}
\end{equation}
 \item 平方完成したい式を式(\ref{135015_17Nov14})の右辺の形式で表現する
 \item $x$の2次の項と1次の項の係数を比較することで、$\Sigma$と$\mu$を求める
\end{itemize}
\end{frame}

\begin{frame}{平方完成}
\begin{itemize}
  \item $\mu_{a|b}, \Sigma_{a|b}$を求める
				\begin{equation}
					p(x_a | x_b) \sim N(x | \mu_{a|b}, \Sigma_{a|b})
					\end{equation}
\end{itemize}
\end{frame}

\begin{frame}{実際にやってみた}
 \begin{itemize}
	\item 条件付きガウス分布に対して適応する
			 \begin{eqnarray*}
				-\frac{1}{2}(x - \mu)^{T}\Sigma^{-1}(x-\mu) &= &
				 -\frac{1}{2}(x_a - \mu_a)^{T}\Sigma_{aa}^{-1}(x_a-\mu_a) \\
				&&-\frac{1}{2}(x_a - \mu_a)^{T}\Sigma_{ab}^{-1}(x_b-\mu_b) \\
				 &&-\frac{1}{2}(x_b - \mu_b)^{T}\Sigma_{ba}^{-1}(x_a-\mu_a) \\
				 &&-\frac{1}{2}(x_b - \mu_b)^{T}\Sigma_{bb}^{-1}(x_b-\mu_b)
			 \end{eqnarray*}
	\item $x_a$についての2次の項を全て取り出すと、
				\begin{equation}
				 -\frac{1}{2}x_a^T\Lambda_{aa}x_a
				\end{equation}
を得る
				\item ここから
							\begin{equation}
							 \Sigma_{a|b} = \Lambda_{aa}
							\end{equation}
 \end{itemize}
\end{frame}

\begin{frame}{平均}
\begin{itemize}
 \item $x_a$についての線形の項をすべて考える
	\begin{equation}
	 x_a^T\{ \Lambda_{aa}\mu_a-\Lambda_{ab}(x_b)-\mu_b\}
	\end{equation}
 \item 一般形についての議論から、この式の$x_a$の係数は$\Sigma^{-1}_{a|b}\mu_{a|b}$と等しくなる
			 \begin{eqnarray}
\mu_{a|b} &=& \Sigma_{a|b}\{\Lambda_{aa}\mu_a-\Lambda_{ab}(x_b-\mu_b)\}\\
				& & \mu_a - \Lambda_{aa}^{-1}\Lambda_{ab}(x_b-\mu_b)
			 \end{eqnarray}
\end{itemize}
\end{frame}

\begin{frame}{精度行列を使わないで求める}

\end{frame}

\begin{frame}{hoge}
 \begin{itemize}
	\item 2つの結果を比べる
				\begin{eqnarray*}
				 \mu_{a|b} &=& \mu_a + \Sigma_{ab}\Sigma_{bb}^{-1}(x_b-\mu_b)\\
				 &=& \mu_a-\Lambda_{aa}^{-1}\Lambda_{ab}(x_b-\mu_b)
				\end{eqnarray*}
			 \begin{eqnarray*}
				\Sigma_{a|b} &=& \Sigma_{aa} - \Sigma_{ab}\Sigma_{bb}^{-1}\Sigma_{ba} \\
& =& \Lambda_{aa}^{-1}
			 \end{eqnarray*}
 \item 条件付き分布$p(x_a|x_b)$は、分割された共分散行列よりも、分割された精度行列を使って表現する方が簡潔
 \end{itemize}
\end{frame}
% \begin{frame}{条件付きガウス分布の平均と共分散}
% \begin{itemize}
%  \item 一般のガウスの指数部分に注目する


% \end{itemize}
% \end{frame}
