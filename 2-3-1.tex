\begin{frame}{ガウス分布の性質}
 \begin{itemize}
  % \item 2つの変数集合の同時分布$p(x_a,x_b)$がガウス分布に従うなら、一方の変数集合が与えられたときの、もう一方の集合の条件付き分布$p(x_a|x_b)$もガウス分布になる
  \item 2つの変数集合の同時分布$p(x_a,x_b)$がガウス分布に従うなら、条件付き分布$p(x_a|x_b),p(x_b|x_a)$もガウス分布になる
  \item さらに、どちらの変数集合の周辺分布$p(x_a),p(x_b)$も同様にガウス分布になる
  \item 2.3.1節と2.3.2節でこれらを示す
 \end{itemize}
\end{frame}

        \begin{frame}{条件付きガウス分布(導入)}
         \begin{itemize}
          \item 多変量ガウス分布の条件付き分布を考える
                \begin{equation}
                 x\sim N(x|\mu, \Sigma)
                \end{equation}
                「$\sim$」はある分布に従う、ということ
          \item この$D$次元ベクトル$x$を2つの互いに素な部分集合$x_a$と$x_b$に分割する
          \item 次式のように、$x_a$は$x$の最初の$M$個の要素で、$x_b$は残りの$D-M$個の要素で構成されるとしても一般性は失わない
                \begin{equation}
                 x =
                  \begin{pmatrix}
                   x_a \\
                   x_b
                  \end{pmatrix}
                \end{equation}

                \begin{equation}
                 \mu =
                  \begin{pmatrix}
                   \mu_a \\
                   \mu_b
                  \end{pmatrix}
                \end{equation}
          \item 共分散行列も同様に与えられる
                \begin{equation}
                 \Sigma =
                  \begin{pmatrix}
                   \Sigma_{aa} & \Sigma_{ab} \\
                   \Sigma_{ba} & \Sigma_{bb}
                  \end{pmatrix}
                \end{equation}
         \end{itemize}
        \end{frame}

        \begin{frame}{精度行列}
         \begin{itemize}
          \item \alert{精度行列}$\Lambda$を導入する
                \begin{equation}
                 \Lambda \equiv \Sigma^{-1}
                \end{equation}

          \item ベクトル$x$の分割に対応する、分割された形式の精度行列を導入する
                \begin{equation}
                 \Lambda=
                  \begin{pmatrix}
                   \Lambda_{aa} && \Lambda_{ab}\\
                   \Lambda_{ba} && \Lambda_{bb}
                  \end{pmatrix}
                \end{equation}
         \end{itemize}
        \end{frame}


          \begin{frame}{この章の目的}
           \begin{itemize}
            \item 条件付きガウス分布もガウス分布に従うことを示す
            \item 条件付きガウス分布は、次式のとおりである
                  \begin{equation}
                   p(x_a | x_b) = \frac{p(x_a, x_b)}{p(x_b)}
                  \end{equation}
                  \begin{itemize}
                   \item $x_b$を観測済の値で固定する
                   \item ここで、$p(x_b)$は正規化するための係数
                  \end{itemize}
            % \item まず$p(x_a,x_b)$の指数部に注目する
            \item まずガウス分布の指数部に注目する
           \end{itemize}
           \begin{equation}
            N(x|\mu,\Sigma) = \frac{1}{(2\pi)^{D/2}}\frac{1}{|\Sigma|^{1/2}}exp\left\{-\frac{1}{2}(x - \mu)^{T}\Sigma^{-1}(x-\mu)\right\}
           \end{equation}
          \end{frame}

            \begin{frame}{条件付き分布の表現}
             \begin{itemize}
              \item ガウス分布の指数部分に注目する
                    \begin{eqnarray*}
                     -\frac{1}{2}(x - \mu)^{T}\Sigma^{-1}(x-\mu) &= &
                      -\frac{1}{2}(x_a - \mu_a)^{T}\Lambda_{aa}^{-1}(x_a-\mu_a) \\
                     &&-\frac{1}{2}(x_a - \mu_a)^{T}\Lambda_{ab}^{-1}(x_b-\mu_b) \\
                     &&-\frac{1}{2}(x_b - \mu_b)^{T}\Lambda_{ba}^{-1}(x_a-\mu_a) \\
                     &&-\frac{1}{2}(x_b - \mu_b)^{T}\Lambda_{bb}^{-1}(x_b-\mu_b)
                    \end{eqnarray*}
              \item $x_a$の関数として見ると、これも二次形式になっている
              \item 対応する条件付き分布$p(x_a | x_b)$もガウス分布となる
             \end{itemize}
            \end{frame}

              \begin{frame}{平方完成}
               \begin{itemize}
                \item ここでの演算を\alert{平方完成}と呼ぶ
                      \begin{equation}
                       -\frac{1}{2}(x-\mu)^{T}\Sigma^{-1}(x-\mu) = -\frac{1}{2}x^T\Sigma^{-1}x+x^T\Sigma^{-1}\mu + const\label{135015_17Nov14}
                      \end{equation}
                \item 平方完成したい式を式(\ref{135015_17Nov14})の右辺の形式で表現する
                \item $x$の2次の項と1次の項の係数を比較することで、$\Sigma$と$\mu$を求める
               \end{itemize}
              \end{frame}

                \begin{frame}{平方完成}
                 \begin{itemize}
                  \item $\mu_{a|b}, \Sigma_{a|b}$を求める
                        \begin{equation}
                         p(x_a | x_b) \sim N(x | \mu_{a|b}, \Sigma_{a|b})
                        \end{equation}
                 \end{itemize}
                \end{frame}

                  \begin{frame}{実際にやってみた}
                   \begin{itemize}
                    \item 条件付きガウス分布に対して適応する
                          \begin{eqnarray*}
                           -\frac{1}{2}(x - \mu)^{T}\Sigma^{-1}(x-\mu) &= &
                            -\frac{1}{2}(x_a - \mu_a)^{T}\Sigma_{aa}^{-1}(x_a-\mu_a) \\
                           &&-\frac{1}{2}(x_a - \mu_a)^{T}\Sigma_{ab}^{-1}(x_b-\mu_b) \\
                           &&-\frac{1}{2}(x_b - \mu_b)^{T}\Sigma_{ba}^{-1}(x_a-\mu_a) \\
                           &&-\frac{1}{2}(x_b - \mu_b)^{T}\Sigma_{bb}^{-1}(x_b-\mu_b)
                          \end{eqnarray*}
                    \item $x_a$についての2次の項を全て取り出すと、
                          \begin{equation}
                           -\frac{1}{2}x_a^T\Lambda_{aa}x_a
                          \end{equation}
                          を得る
                    \item ここから
                          \begin{equation}
                           \Sigma_{a|b} = \Lambda_{aa}
                          \end{equation}
                   \end{itemize}
                  \end{frame}

                    \begin{frame}{平均}
                     \begin{itemize}
                      \item $x_a$についての線形の項をすべて考える
                            \begin{equation}
                             x_a^T\{ \Lambda_{aa}\mu_a-\Lambda_{ab}(x_b)-\mu_b\}
                            \end{equation}
                      \item 一般形についての議論から、この式の$x_a$の係数は$\Sigma^{-1}_{a|b}\mu_{a|b}$と等しくなる
                            \begin{eqnarray}
                             \mu_{a|b} &=& \Sigma_{a|b}\{\Lambda_{aa}\mu_a-\Lambda_{ab}(x_b-\mu_b)\}\\
                             & & \mu_a - \Lambda_{aa}^{-1}\Lambda_{ab}(x_b-\mu_b)
                            \end{eqnarray}
                     \end{itemize}
                    \end{frame}

                      \begin{frame}{精度行列を使わないで求める}

                      \end{frame}

                        \begin{frame}{hoge}
                         \begin{itemize}
                          \item 2つの結果を比べる
                                \begin{eqnarray*}
                                 \mu_{a|b} &=& \mu_a + \Sigma_{ab}\Sigma_{bb}^{-1}(x_b-\mu_b)\\
                                 &=& \mu_a-\Lambda_{aa}^{-1}\Lambda_{ab}(x_b-\mu_b)
                                \end{eqnarray*}
                                \begin{eqnarray*}
                                 \Sigma_{a|b} &=& \Sigma_{aa} - \Sigma_{ab}\Sigma_{bb}^{-1}\Sigma_{ba} \\
                                 & =& \Lambda_{aa}^{-1}
                                \end{eqnarray*}
                          \item 条件付き分布$p(x_a|x_b)$は、分割された共分散行列よりも、分割された精度行列を使って表現する方が簡潔
                         \end{itemize}
                        \end{frame}
                          % \begin{frame}{条件付きガウス分布の平均と共分散}
                            % \begin{itemize}
                              %  \item 一般のガウスの指数部分に注目する


                              % \end{itemize}
                            % \end{frame}
