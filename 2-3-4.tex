\begin{frame}
 \begin{itemize}
	\item  ある多変量ガウス分布から、観測値$\{x_n\}$が独立に得られたと仮定したデータ集合
				 \begin{equation}
					X=(x_1,...,x_n)^T
				 \end{equation}
				 があるとき、その分布のパラメータは最尤推定法で推定できる
	\item 尤度関数は、
				\begin{equation}
				 \ln  p(X|\mu, \Sigma) = -\frac{ND}{2}\ln (2\pi)-\frac{N}{2}\ln |\Sigma|-\frac{1}{2}\sum_{n=1}^{N}(x_n-\mu)^T\Sigma^{-1}(x_n-\mu)
\end{equation}
	\item これを整理すると、尤度関数は次の2つの量によってのみ依存していることが分かる
				\begin{eqnarray}
				 \sum_{n=1}^{N}x_n\\
				 \sum_{n=1}^{N}x_nx_n^T
				\end{eqnarray}
	\item これらをガウス分布の十分統計量という
 \end{itemize}
\end{frame}

\begin{frame}{結果}
 \begin{itemize}
	\item 結果は次のとおり
				\begin{eqnarray}
				\mu_{ML} &=& \frac{1}{N}\sum_{n=1}^{N}x_n\\
				 \Sigma_{ML}=\frac{1}{N}\sum_{n=1}^{N}(x_n-\mu_{ML})(x_n-\mu_{ML})^T
				\end{eqnarray}
 \end{itemize}
\end{frame}

\begin{frame}{最尤推定解の期待値}
\begin{itemize}
 \item 次のようになる
			 \begin{eqnarray}
				E[\mu_{ML}]&=&\mu\\
				E[\Sigma_{ML}]=\frac{N-1}{N}\Sigma
			 \end{eqnarray}
 \item 偏りがあるが、これは別の推定量$\Sigma$
			 \begin{equation}
				\Sigma = \frac{1}{N-1}\sum_{n=1}^{N}(x_n-\mu_{ML})(x_n-\mu_{ML})^T
			 \end{equation}
			 を定義することで補正することができる

\end{itemize}

\end{frame}
