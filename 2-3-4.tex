\begin{frame}{2.3.4 ガウス分布の最尤推定}
 \begin{itemize}
  \item  ある多変量ガウス分布から、観測値$\{x_n\}$が独立に得られたと仮定したデータ集合
         \begin{equation}
          X=(x_1,...,x_n)^T
         \end{equation}
         がある時その分布のパラメータは最尤推定法で推定できる
  \item 尤度関数は、
        \begin{equation}
         \ln  p(X|\mu, \Sigma) = -\frac{ND}{2}\ln (2\pi)-\frac{N}{2}\ln |\Sigma|-\frac{1}{2}\sum_{n=1}^{N}(x_n-\mu)^T\Sigma^{-1}(x_n-\mu)
        \end{equation}
  \item これを整理すると、尤度関数は次の2つの量によってのみ依存していることが分かる
        \begin{eqnarray}
         \sum_{n=1}^{N}x_n, \ \ \  \sum_{n=1}^{N}x_nx_n^T
        \end{eqnarray}
  \item これらをガウス分布の\alert{十分統計量}という
 \end{itemize}
\end{frame}

\begin{frame}{最尤推定解}
 \begin{itemize}
  \item 最尤推定解は次のとおり
        \begin{eqnarray}
         \mu_{ML} &=& \frac{1}{N}\sum_{n=1}^{N}x_n\\
         \Sigma_{ML}&=&\frac{1}{N}\sum_{n=1}^{N}(x_n-\mu_{ML})(x_n-\mu_{ML})^T
        \end{eqnarray}
 \end{itemize}
\end{frame}

\begin{frame}{最尤推定解の期待値}
 \begin{itemize}
  \item 真の分布の下での最尤推定解の期待論を評価すると、次の結果を得る(演習2.35)
        \begin{eqnarray}
         \mathbb{E}[\mu_{ML}]&=&\mu\\
         \mathbb{E}[\Sigma_{ML}]&=&\frac{N-1}{N}\Sigma
        \end{eqnarray}
        \begin{itemize}
         \item 平均についての最尤推定量の期待論は真の平均に等しい
         \item 共分散の最尤推定量の期待論は真の値より小さいが、これは別の推定量$\widetilde{\Sigma}$
               \begin{equation}
                \widetilde{\Sigma} = \frac{1}{N-1}\sum_{n=1}^{N}(x_n-\mu_{ML})(x_n-\mu_{ML})^T
               \end{equation}
               を定義することで補正することができる
        \end{itemize}
 \end{itemize}
\end{frame}
