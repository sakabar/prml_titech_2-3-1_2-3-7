\begin{frame}{周辺ガウス分布}
\begin{itemize}
 \item 同時分布$p(x_a,x_b)$がガウス分布であれば、条件付き分布$p(x_a|x_b)$もガウス分布になることを示した。
 \item この周辺分布
\begin{equation}
 p(x_a) = \int p(x_a,x_b)dx_b
\end{equation}
			 がガウス分布になることを示す
\end{itemize}
\end{frame}

\begin{frame}{ho}
\begin{itemize}
 \item $x_b$に関係した項を処理してから、積分を容易にするために平方完成する
\item $x_b$を含む項を取り出すと
			\begin{eqnarray}
			 &&-\frac{1}{2}x^T_b\Lambda_{bb}x_b+x^T_bm \\
				&=& -\frac{1}{2}(x_b-\Lambda_{bb}^{-1}m)^T\Lambda_{bb}(x_b-\Lambda_{bb}^{-1}m) + \frac{1}{2}m^T\Lambda_{bb}^{-1}m \\
				m &=& \Lambda_{bb}\mu_b - \Lambda_{ba}(x_a-\mu_a)
			\end{eqnarray}
\end{itemize}
\end{frame}

\begin{frame}
 \begin{itemize}
	\item $x_b$に依存する部分を取り出して積分する
				\begin{eqnarray}
				 \int exp\{-\frac{1}{2}(x_b-\Lambda_{bb}^{-1}m)T\Lambda_{bb}(x_b-\Lambda_{bb}^{-1}m)\}dx_b
				\end{eqnarray}
	\item この積分は正規化されていないガウス分布なので、正規化係数の逆数になる。
	\item この係数は、平均とは独立で、共分散行列のみに依存する
 \end{itemize}
\end{frame}


\begin{frame}{途中計算}
\begin{itemize}
 \item とりあえず省く
\end{itemize}
\end{frame}

\begin{frame}{結論}
\begin{itemize}
 \item 周辺分布$p(x_a)$の平均と共分散は次のようになる
			\begin{eqnarray}
			 E[x_a] &=&  \mu_a\\
			 cov[x_a]&=&\Sigma_{aa}
			\end{eqnarray}
 \item 分割された共分散行列について簡潔に表現される
			 \begin{itemize}
				\item 条件付き分布のときと対照的
			 \end{itemize}
\end{itemize}
\end{frame}
