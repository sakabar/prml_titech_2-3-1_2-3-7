\begin{frame}{2.3.2 周辺ガウス分布}
 \begin{itemize}
  \item 同時分布$p(x_a,x_b)$がガウス分布であれば、条件付き分布$p(x_a|x_b)$もガウス分布になることを示した。
  \item この周辺分布
        \begin{equation}
         p(x_a) = \int p(x_a,x_b)dx_b
        \end{equation}
        がガウス分布になることを示す
  \item ここでも同時分布の指数部分の二次形式に注目し、周辺分布$p(x_a)$の平均と共分散を特定することで効率的に計算できる
 \end{itemize}
\end{frame}


\begin{frame}{計算の流れ}
 \begin{itemize}
  \item 指数部の$x_b$に関係した項を処理してから、積分を容易にするために平方完成する
        \begin{eqnarray*}
         \int p(x_a,x_b)dx_b &=& \int \frac{1}{C_1}exp\left\{x_b^T\Lambda x_b+x_a^T\Lambda x_a\right\}dx_b \\
         &= &  \frac{1}{C_1}exp(x_a^T\Lambda x_a)\int exp\left\{x_b^T\Lambda x_b\right\}dx_b \\
         & =& \frac{1}{C_1}exp(x_a^T\Lambda x_a)\cdot C_2
        \end{eqnarray*}
 \end{itemize}
\end{frame}


\begin{frame}{計算}
 \begin{itemize}
  \item 指数部の$x_b$に関係した項を処理してから、積分を容易にするために平方完成する
  \item $x_b$を含む項を取り出すと
        \begin{eqnarray*}
         && -\frac{1}{2}(x - \mu)^{T}\Sigma^{-1}(x-\mu) \\
         &=&-\frac{1}{2}x^T_b\Lambda_{bb}x_b+x^T_bm \\
         &=& -\frac{1}{2}(x_b-\Lambda_{bb}^{-1}m)^T\Lambda_{bb}(x_b-\Lambda_{bb}^{-1}m) + \frac{1}{2}m^T\Lambda_{bb}^{-1}m
        \end{eqnarray*}
        ただし、
        \begin{equation*}
         m =  \Lambda_{bb}\mu_b - \Lambda_{ba}(x_a-\mu_a)
        \end{equation*}
 \end{itemize}
\end{frame}

\begin{frame}{積分}
 \begin{itemize}
  \item 指数部の式は次のとおり
        \begin{equation}
         -\frac{1}{2}(x_b-\Lambda_{bb}^{-1}m)^T\Lambda_{bb}(x_b-\Lambda_{bb}^{-1}m) + \frac{1}{2}m^T\Lambda_{bb}^{-1}m
        \end{equation}
        \begin{itemize}
         \item 右辺第1項はガウス分布の標準的な二次形式
         \item 残りの項は$x_b$に依存しない
        \end{itemize}
  \item $x_b$に関係しない部分を無視して考え、後で正規化係数を求めてつじつまを合わせる
        % \begin{equation}
        %  \int exp(f(x)+c)dx = exp(c)\int exp(f(x))dx
          % \end{equation}
 \end{itemize}
\end{frame}


\begin{frame}{途中計算}
 \begin{itemize}
  \item この二次形式の指数を取り、$x_b$で積分する
        % $x_b$に依存する部分を取り出して積分する
        \begin{eqnarray}
         \int exp\{-\frac{1}{2}(x_b-\Lambda_{bb}^{-1}m)^T\Lambda_{bb}(x_b-\Lambda_{bb}^{-1}m)\}dx_b
        \end{eqnarray}
  \item この積分は正規化されていないガウス分布なので、正規化係数の逆数になる。
  \item ガウス分布の正規化係数は平均とは独立で、共分散行列のみに依存するため、この積分も共分散行列のみに依存する
 \end{itemize}
\end{frame}

\begin{frame}{結論}
 \begin{itemize}
  \item 周辺分布$p(x_a)$の平均と共分散は次のようになる
        \begin{eqnarray}
         E[x_a] &=&  \mu_a\\
         cov[x_a]&=&\Sigma_{aa}
        \end{eqnarray}
  \item 分割された共分散行列について簡潔に表現される
        \begin{itemize}
         \item 条件付き分布のときと対照的
        \end{itemize}
 \end{itemize}
\end{frame}
