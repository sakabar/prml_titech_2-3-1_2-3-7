\begin{frame}{2.3.5 逐次推定}
 \begin{itemize}
  \item 逐次的な方法では、データ点を一度に1つずつ処理しては、それを廃棄する
  \item オンラインな応用分野や、すべてのデータ点を一度に一括処理することが不可能な大規模データ集合を扱う場合に重要
  \item まずは、平均の最尤推定量$\bm{\mu}_{ML}$について考える
 \end{itemize}
\end{frame}

\begin{frame}{平均の最尤推定量の逐次推定}
 \begin{itemize}
  \item $\bm{\mu}_{ML}^{(N)}$を変形すると、次のようになる
        \begin{eqnarray}
         \bm{\mu}_{ML}^{(N)} &= &\frac{1}{N}\sum_{n=1}^{N}\bm{x}_n \nonumber \\
         & =& \frac{1}{N}\bm{x}_N+\frac{1}{N}\sum_{n=1}^{N-1}\bm{x}_n \nonumber \\
         & =& \frac{1}{N}\bm{x}_N + \frac{N-1}{N}\bm{\mu}_{ML}^{(N-1)}\nonumber \\
         & =& \bm{\mu}_{ML}^{(N-1)} +\frac{1}{N} (\bm{x}_N-\bm{\mu}_{ML}^{(N-1)})\label{162008_21Nov14}
        \end{eqnarray}
 \end{itemize}
\end{frame}

\begin{frame}{逐次推定}
 \begin{itemize}
  \item この結果は次のように分かりやすく解釈できる
        \begin{eqnarray}
         \bm{\mu}_{ML}^{(N)} &= &\frac{1}{N}\sum_{n=1}^{N}\bm{x}_n \\
         & =& \bm{\mu}_{ML}^{(N-1)} +\frac{1}{N} (\bm{x}_N-\bm{\mu}_{ML}^{(N-1)})
        \end{eqnarray}
  \item $N-1$個のデータを観測した時点で、$\bm{\mu}$の推定値は$\bm{\mu}_{ML}^{(N-1)}$となっている。
  \item ここで、データ点$\bm{x}_N$を観測すると、$1/N$に比例する小さな量だけ「誤差信号」$(\bm{x}_N-\bm{\mu}_{ML}^{(N-1)})$の方へ、古い推定量を移動させて推定量$\bm{\mu}_{ML}^{(N)}$を修正する
  \item Nが増えるにつれて、後続のデータ点からの影響はより小さくなる
 \end{itemize}
\end{frame}

\begin{frame}{汎用的な逐次学習の定式化}
 \begin{itemize}
  \item 先の例では、全体をまとめてバッチ処理する式と逐次推定する式が等しいので、明らかに同じ解が得られる
  \item しかしこの方法で逐次アルゴリズムを導出することが、いつもできるわけではない
  \item \alert{Robbins-Monroアルゴリズム}を導入する
 \end{itemize}
\end{frame}


%%TODO 図2.10の挿入
\begin{frame}{準備}
 \begin{itemize}
  \item 同時分布$p(z,\theta)$に従う確率変数$\theta$と$z$の対を考える
  \item $\theta$が与えられたときの$z$の条件付き期待値によって、決定論的な関数$f(\theta)$を定義する
        \begin{equation}
         f(\theta) \equiv \mathbb{E}[z|\theta] = \int zp(z|\theta)dz
        \end{equation}
        \begin{itemize}
         \item このように定義された関数を回帰関数と呼ぶ
        \end{itemize}
  \item ここでの目標は、$f(\theta^\ast)=0$の根$\theta^\ast$を求めること
 \end{itemize}
\end{frame}

\begin{frame}{仮定}
 \begin{itemize}
  \item 次のような仮定を置く
        \begin{eqnarray}
         \mathbb{E}[(z-f)^2|\theta] &<& \infty\\
         \theta^{(N)}&=& \theta^{(N-1)}-a_{N-1}z(\theta^{(N-1)})
        \end{eqnarray}
  \item ただし、$z(\theta^{(N)})$は$\theta$が値$\theta^{(N)}$を取るときに観測される$z$の値
  \item 係数$\{a_N\}$は以下の条件を満たす正数の系列
        \begin{equation}
         \lim_{N \rightarrow \infty}a_N=0
        \end{equation}
        \begin{itemize}
         \item この過程が極限値に収束できるように、解の逐次的な修正量を減らすことを保証
        \end{itemize}
        \begin{equation}
         \sum_{N=1}^{\infty}a_N=\infty
        \end{equation}
        \begin{itemize}
         \item アルゴリズムが根以外に速すぎる収束をしないことを保証
        \end{itemize}
        \begin{equation}
         \sum_{N=1}^{\infty}a_N^2 < \infty
        \end{equation}
        \begin{itemize}
         \item 蓄積されたノイズの分散を有限に抑え、収束を阻害しないことを保証
        \end{itemize}
 \end{itemize}
\end{frame}

\begin{frame}{Robbins-Monroアルゴリズム}
 \begin{itemize}
  \item 定義より、最尤推定解$\theta_{ML}$は負の対数尤度関数の停留点であるため、
        \begin{equation}
         -\frac{\partial }{\partial \theta}\left\{\left.\frac{1}{N}\sum_{n=1}^{N}\ln  p(x_n|\theta)\right\}\right|_{\theta_{ML}} = 0
        \end{equation}
  \item 微分と総和の演算を交換し、$N\rightarrow\infty$の極限を考えると次の式を得る
        \begin{equation}
         -\lim_{N \rightarrow \infty}\frac{1}{N}\sum_{n=1}^{N}\frac{\partial}{\partial \theta}\ln p(x_n|\theta)=\mathbb{E}_x\left[-\frac{\partial}{\partial \theta}\ln p(x|\theta)\right]
        \end{equation}
 \end{itemize}
\end{frame}

\begin{frame}{Robbins-Monro手続きの適用}
 \begin{itemize}
  \item 最尤推定解を求めることは、回帰関数の根を求めることに相当することがわかる
  \item ゆえに、次の形でRobbins-Monro手続きを適用できる
        \begin{equation}
         \theta^{(N)}=\theta^{(N-1)}-a_{N-1}\frac{\partial}{\partial \theta^{(N-1)}}[-\ln p(x_N|\theta^{(N-1)})]\label{161911_21Nov14}
        \end{equation}
 \end{itemize}
\end{frame}

\begin{frame}{ガウス分布への適用}
 \begin{itemize}
  \item パラメータ$\theta^{(N)}$はガウス分布の平均の推定量$\mu_{ML}^{(N)}$であり、確率変数$z$は
        \begin{equation}
         z=-\frac{\partial}{\partial \mu_{ML}} \ln p(x|\mu_{ML}, \sigma^2) = -\frac{1}{\sigma^2}(x-\mu_{ML})\label{161904_21Nov14}
        \end{equation}
  \item 式(\ref{161904_21Nov14})を式(\ref{161911_21Nov14})に代入し、係数$a_N$を$a_N=\sigma^2/N$となるように選ぶと、式(\ref{162008_21Nov14})の1変数の形式のものが得られる
 \end{itemize}
\end{frame}
